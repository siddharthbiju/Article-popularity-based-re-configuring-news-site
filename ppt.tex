\documentclass{beamer}
\usepackage[utf8]{inputenc}
\usetheme{PaloAlto}

\title{Automatic reconfiguring news site}
\begin{document}
	\begin{frame}{Page of Contents}
		\tableofcontents
	\end{frame}
	\section{Intro}
	\begin{frame}{Introduction}
		\begin{itemize}
			\item  News is the information about an event happening right now.
			\item Online news reading has become a popular way to read news articles from a huge collection of news sources all around.
			\item In this work we present a method for identifying top news in the web environment that consists of diversified news portals. 
		\end{itemize}
	\end{frame}
	\section{Statement}
	\begin{frame}{Problem Statement}
		 News websites are one of the most visited destinations on the web. As there are many news portals created on a daily basis, each having its own preference for which news are important, detecting unbiased important and popular news are the complex part of the news portal development.
	\end{frame}
	\section{Approaches}
	\begin{frame}{Approach 1}
		\footnotesize
	Input : nil \\
	Output :  website with trending news
	
	\footnotesize
	\textbf{server side algorithm: }\\
	step 1 :  for each 24 hours : \\
	\hspace{.2cm} step 1.1 : twurl /1.1/trends/place.json?id=1 > copy.json\\
	step 2:  data = file.read() \\
	step 3:  from i=0 to i=10 \\
	\hspace{.2cm}step 3.1: temp = data[0]["trends"][i]["name"]\\   
	\hspace{.2cm}Step 3.2:  temp = temp[1:]  \\
\hspace{.2cm}Step 3.3:  temp = temp + " latest news articles"\\
\hspace{.2cm}Step 3.4:  keyarray.append(temp)  \\
step 4: crawl google for news links using keyarray values\\
step 5: seperate title and content from links obtained  and append to an array with id \\
step 6: accept get request with news id\\
step 7: return array[id]
		
	\end{frame}
	\begin{frame}{Approach 1 cont ...}	
	\footnotesize
	\textbf{client side algorithm: }\\
	step 1 : render  navbar .js  \\	
	step 2:  render  contents .js \\
	\vspace{5mm}
	\textbf{contents .js}\\
	\vspace{5mm}
	1. for(i=0;i<10;i++)\\
	\hspace{.2cm}1.1 get 127.0.0.1 with i as id\\
	\hspace{.2cm}1.2 store response in an array\\
	2. render title and contents in array
	
  
\end{frame}
	\begin{frame}{Approach 2}
	\footnotesize
	Input: nil \\
	Output: website with popular news articles\\
	\vspace{.2cm}
	\footnotesize
	step 1: For every 1 hour extract latest news article details from different\\ \hspace{1.2mm} news site 
	(includes TimeofPublish ,ups,downs)\\
	step 2:Find the number of articles extracted\\
	step 3: Using reddit hot ranking algorithm sort the news articles based on popularity rank\\
	step 4:Display each article based on the rank calculated.\\
	\vspace{0.5cm}
	\end{frame}
	\begin{frame}{Approach 2 cont ...}
	\textbf{reddit hot ranking algorithm:}\\
	step 1:start\\
	step 2:for each article extract the time of publish,upcount,downcount\\
	step 3:set ep=1970,1,1 in datetime format\\
	step 4.define a function epsec(date)\\
	\hspace{.3cm}step 4.1:td=date - ep\\
	\hspace{.3cm}step 4.2:return td.days * 86400 + td.seconds +\\ \hspace{.8cm}(float(td.microseconds) / 1000000)\\
	step 5:define a function score(upcount,downcount)\\
	\hspace{.3cm}step 5.1:return upcount-downcount\\
	step 6:define a function hot(upcount,downcount,date)\\
    \hspace{.3cm}step 6.1:s = score(upcount, downcount)\\
    \hspace{.3cm}step 6.2:order = log(max(abs(s), 1), 10)\\
    \hspace{.3cm}step 6.3:sign = 1 if s is grater than 0 else -1 if s less than 0 else 0\\
    \end{frame}
    \begin{frame}{Approach 2 cont ...}
    
    \hspace{.3cm}step 6.4:seconds = epsec(date) - 1134028003\\
    \hspace{.3cm}step 6.5:return round(sign * order + seconds / 45000, 7)\\
    step 7:stop
    \end{frame}
	
	
	
	       
	
	\begin{frame}{Method 3}
		\begin{itemize}
			\item Algorithm
		\end{itemize}
		Step 1: Read dataset containing articles \newline
		Step 2: Extract title,Number of articles n \newline
		Step 3: Do steps 4,5 for n articles \newline
		Step 4: Search title using any search engine \newline
		Step 5: Read and store corresponding number of hits \newline
		Step 6: sort articles according to number of hits \newline
		Step 6: Do steps 7 for n articles \newline
		Step 7: Print title and content of sorted articles
		
	\end{frame}
	\begin{frame}{Approach 4}
		   \begin{itemize}
        	\item Algoritm\\
        	\footnotesize
        	Input: News title\\
        	Output: Articles sorted based on popularity \\  
   	   \end{itemize}	
            Step 1:Search title using search engine\\
            Step 2:Repeat Step 3,4 until no more chances left\\
            Step 3:Views of every page is stored\\
            Step 4:Compare views of all pages\\
            Step 5:Content of page with high views is extracted\\
            Step 6:Repeat from Step 1 until no more input\\
            Step 7:All articles extracted are sorted according to the number of views\\
            Step 8:Print title and content of sorted articles\\


	\end{frame}
	\section{Comparison}
	\begin{frame}{Comparison}
	\begin{itemize}
            \item On comparing Approach 1 and 4 ,Approach 4 takes more time to complete as it searches entire pages available for satisfying the title inputed.      
            \item comparing Approach 1 and 2,Approach 2 its more difficult to implement to news articles as downvote and upvote my not be present in all articles.
  	    \end{itemize}

	\end{frame}
	\section{Ethical and Social Relevance}
	\begin{frame}{Relevance}
	\end{frame}
	\section{Analysis}
	\begin{frame}{Problem analysis}
	\end{frame}
	\section{Solution}
	\begin{frame}{Solution}
	\end{frame}
	\section{Conclusion}
	\begin{frame}{Conclusion}
	\end{frame}
	\section{Reference}
	\begin{frame}{Reference}
	\end{frame}
\end{document}
